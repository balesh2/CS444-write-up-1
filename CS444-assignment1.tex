\documentclass[letterpaper,10pt]{article}

\usepackage{geometry}
\usepackage{hyperref}
\geometry{textheight=8.5in, textwidth=6in}

\newcommand{\toc}{\tableofcontents}

\def\name{Helena A. Bales}
\title{CS 444 Project 1: Getting Acquainted}
\author{Helena A. Bales}

%% The following metadata will show up in the PDF properties
% \hypersetup{
%   colorlinks = false,
%   urlcolor = black,
%   pdfauthor = {\name},
%   pdfkeywords = {cs444 ``operating systems''},
%   pdftitle = {CS 444 Project 1: Getting Acquainted},
%   pdfsubject = {CS 444 Project 1},
%   pdfpagemode = UseNone
% }

\parindent = 0.0 in
\parskip = 0.1 in

\begin{document}
\maketitle

\begin{abstract}
This document contains a log of commands used to complete the qemu orientation
given in the project description, and explanation of every flag in the listed
qemu command line, answers to questions related to the concurrency exercise, a
version control log formatted as a table, and a work log. These documents form
the solution to project 1. The purpose of this project is to get acquainted with
the tools used throughout Operating Systems II, and to solve the concurrency
programming problem. The concurrency problem allows for the development of
thinking in parallel, and for practicing threads and programming C and ASM. The
commands used to start a linux virtual machine serves as practice using qemu and
building the Linux kernel. This assignment shall serve as a foundation for the
skills required for the Operating Systems II course.
\end{abstract}

\clearpage

\tableofcontents

\clearpage

\section{Section 1: Command Log}
The commands used to perform the actions requested by the project description
for Project 1 of Operating Systems II.  The actions comprise of setting up and
starting the virtual machine using qemu, changing the config file, building the
linux kernel, and starting the virtual machine using the new kernel.

\subsection{Commands to Set Up and Launch Qemu}
\begin{itemize}
\item ssh balesh@os-class.engr.oregonstate.edu
\item mkdir /scratch/fall2016/cs444-026
\item cd /scratch/fall2016/cs444-026
\item git init
\item git clone git://git.yoctoproject.org/linux-yocto-3.14
\item git remote add origin https://github.com/balesh2/CS444-1.git
\item git add --all
\item git commit
\item git push origin master
\item cp /scratch/fall2016/files/* ./
\item cp /scratch/opt/environment-setup-i586-poky-linux.csh ./
\item source ./environment-setup-i586-poky-linux.csh
\item mv ./config-3.14.26-yocto-qemu ./.config
\item qemu-system-i386 -gdb tcp::5526 -S -nographic -kernel bzImage-qemux86.bin -drive \-file=core-image-lsb-sdk-qemux86.ext3,if=virtio -enable-kvm -net none -usb \- -localtime --no-reboot --append "root=/dev/vda rw console=ttyS0 debug".
\item gdb
\begin{itemize}
\item target remote :5526
\item continue
\end{itemize}
\end{itemize}

\subsection{Commands to Build Kernel and Launch Qemu}
\begin{itemize}
\item cd /scratch/fall2016/cs444-026/
\item cp /scratch/fall2016/files/config-3.14.26-yocto-qemu ./.config
\item make -j4 all
\item source ./environment-setup-i586-poky-linux.csh
\item qemu-system-i386 -gdb tcp::5526 -S -nographic -kernel bzImage-qemux86.bin -drive \-file=core-image-lsb-sdk-qemux86.ext3,if=virtio -enable-kvm -net none -usb \- -localtime --no-reboot --append "root=/dev/vda rw console=ttyS0 debug".
\item gdb
\begin{itemize}
\item target remote :5526
\item continue
\end{itemize}
\end{itemize}


\section{Section 2: Explanation of Qemu Flags}
The flags for the following command will be explaned: \newline
qemu-system-i386 -gdb tcp::5526 -S -nographic -kernel bzImage-qemux86.bin -drive
file=core-image-lsb-sdk-qemux86.ext3,if=virtio -enable-kvm -net none -usb
-localtime --no-reboot --append "root=/dev/vda rw console=ttyS0 debug".\newline

\subsection{Explanation of All Qemu Command Line Flags}
\begin{itemize}
\item -gdb : Blocks the boot sequence until continued from GDB.
\item tcp::5526 : Sets the TCP port number to my assigned port.
\item -S : Do not start CPU at startup. Start CPU by typing 'c' in the monitor.
\item -nographic : Totally disable visuals so that qemu is a simple command line
application.
\item -kernel bzImage-qemux86.bin : Specifies the kernel image to be used.
\item -drive file=core-image-lsb-sdk-qemux86.ext3,if=virtio : Opens the specific
kernel image file.
\item -enable-kvm : Enabel KVM full virtualization support.
\item -net none : Does not create a Network Interface Card.
\item -usb : Enable the USB driver.
\item -localtime : Sets time to local computer time.
\item --no-reboot : Exit instead of rebooting.
\item --append "root=/dev/vda rw console=ttyS0 debug". : Sets root and
read/write settings.
\end{itemize}

\section{Section 3: Concurrency Description}
\subsection{Purpose of Assignment}
The purpose of this assignment is to create a simple program for concurrency
synchronization. This is important because creating a solution to this problem
requires parallel thinking, which is an important skill to have in industry.
This problem presents a possible solution to the event handler issue where one
buffer contains items to be added or removed by threads. In this problem,
consumers remove items from the buffer and producers create items and add them
to the buffer. When trying to perform both actions of adding to and removing
from the buffer, the buffer must be locked in order to insure that only one
thread can make changes at once to avoid conflicts.
\subsection{Approach to Problem}
My approach to this problem is to generate the consumer threads in main using a
loop. Each consumer is given a value in the range of 0 to 31. Mutex is used for
synchronization. The consumer locks the mutex and modifies the buffer. After
making modifications it removes the locks and sleeps, then prints the values.
The producer is implemented using a loop in main, from 0 to 31. It takes the
mutex locks and fills in one of the elements in the buffer. After that the lock
is removed and it sleeps before printing and repeating, similar to the process
for the consumer.
\subsection{Testing of Solution}
I tested the solution by providing a variety of inputs and checking that the
output matched the expected. I gave it various numbers of consumers to create,
in the range of 0 to 31, and values greater than 31.
\subsection{Learning Outcomes}
Through this assignment I learned about the need for locking the buffer when it
is being modified. This is important since it is in flux when a thread is
modifying it. Since this is the case, it can only be modified by one thread at
once. Additionally, I was reminded of how to create threads in C. I learned how
to use Mutex to lock.

\section{Section 4: Version Control Logs}
\begin{tabular}{l l l}\textbf{Detail} & \textbf{Author} & \textbf{Description}\\\hline
\href{https://github.com/balesh2/CS444-write-up-1/commit/42d491ee4c85c1fb33946119ef4ccd85bc7a22d9}{42d491e} & balesh2 & create tex and c program\\\hline
\href{https://github.com/balesh2/CS444-write-up-1/commit/a54fb1bd1cbc1bba8d00c19603d0c71cd2b341ce}{a54fb1b} & balesh2 & update write up\\\hline
\href{https://github.com/balesh2/CS444-write-up-1/commit/66a128b6cb787630ca5ee018be159e8916a0d639}{66a128b} & balesh2 & finish concurrency problem\\\hline
\href{https://github.com/balesh2/CS444-write-up-1/commit/0a30e82f7164fd99630aa3bc042ef80799391d23}{0a30e82} & balesh2 & add makefile, script to generate commit log\\\hline
\href{https://github.com/balesh2/CS444-write-up-1/commit/15a674b748965d1b7e9fb3f91cc396f0b858b991}{15a674b} & balesh2 & finishes makefile\\\hline
\href{https://github.com/balesh2/CS444-write-up-1/commit/8f486187f63e041b5ab9d1912ca149940b585dc7}{8f48618} & balesh2 & updates commit log\\\hline
\href{https://github.com/balesh2/CS444-write-up-1/commit/2c55e427910e12c9c37a1c0a8d59a05bddd97c35}{2c55e42} & balesh2 & update makefile\\\hline\end{tabular}


\section{Section 5: Work Log}
\subsection{October 9th, 2016 - 3:00 pm}
Created folder on class server called cs444-026.

Cloned kernel from given repository.
\subsection{October 10th, 2016 - 10:00 am}
Set up environment configuration.

Copyied files from /scratch/fall2016/files/.

Started up VM.

Copyied file from /scratch/fall2016/files/config-3.14.26-yocto-qemu.

Built Linux kernel.

Started VM with new kernel.
\subsection{October 10th, 2016 - 3:30 pm}
Started \LaTeX write-up of assignment.

Started concurrency program.
\subsection{October 10th, 2016 - 5:15 pm}
Finished write up except for concurrency problem and version control log.
\subsection{October 10th, 2016 - 6:30 pm}
Finish concurrency code except one bug.

Finish concurrency write up.
\subsection{October 10th, 2016 - 7:00 pm}
Fixed last bug in concurrency program.

Generated git commit log using script from online.

Created Makefile.


\end{document}
